% \documentclass[10pt]{beamer}
% \usetheme{Berkeley}

%%%%%%%%%%%%%%%%%%%%%%%%%%%%%%%%%%%%%
%
\documentclass[xcolor=dvipsnames,hyperref={breaklinks=true},mathserif,
professionalfont,12pt]{beamer} 
%\usecolortheme[named=RoyalBlue]{structure} 
\useoutertheme{infolines} 
\usetheme[]{Pittsburgh} 
\setbeamertemplate{items}[ball] 
\setbeamertemplate{blocks}[rounded][shadow=true]
\setbeamertemplate{navigationsymbols}{}
\usefonttheme{serif}
\usefonttheme{professionalfonts}
\setbeamerfont{frametitle}{size=\normalsize}
\setbeamerfont{text}{size=\small}
\author{Cyrille Reggio}
%
%\usepackage{amsmath,amssymb,amsthm,amssymb}
%\usepackage{mathtools}
%\usepackage{bb
%\usepackage{graphicx}
%\usepackage{float}
%\usepackage{ucs}
%\usepackage{txfonts}
%\usepackage{subfigure}
%\usepackage{color}
%\usepackage{xcolor}
%\usepackage{listings}
%\usepackage{time}
%
%%%%%%%%%%%%%%%%%%%%%%%%%%%%%%%%%%%

\setbeamertemplate{items}[ball] 
\usepackage[utf8]{inputenc}
\usepackage[french]{babel}
\usepackage[T1]{fontenc}
\usepackage{amsmath}
\usepackage{amsfonts}
\usepackage{amssymb}
\usepackage{graphicx}
\usepackage{soul}
\usepackage{tabu}
%\usetheme{default}
\usepackage{makeidx} %obligé dans beamer
\usepackage{float}
\usepackage{mathtools}
\usepackage{wrapfig}
\usepackage{ragged2e}
\usepackage{gensymb} 
\usepackage{fancyhdr}
\usepackage{natbib}
\usepackage[export]{adjustbox}

\usepackage{pdfpages}
\usepackage{supertabular}

\usepackage{array}
\usepackage{url}
%\usepackage{enumitem} Sinon les points disparaissent des itemize !!!
\usepackage{subcaption}
\usepackage{caption}
%\captionsetup{justification=raggedright,singlelinecheck = false}
\usepackage{multicol}
\usepackage{newcent}
\usepackage{environ}
\usepackage{emptypage}

\logo{\includegraphics[scale=0.15]{img/enteteENSM.png}} 
\usepackage{makeidx} 
\makeindex
%%%%%%%%%%%%ndex%%%%%%%%%%%%%%%%%%%%%%%%%%%%%%%%%%%%%%%%%%
\NewEnviron{myequation}{%
\begin{equation}
\scalebox{1.5}{$\BODY$}
\end{equation}
}
\newenvironment{conditions}
  {\par\vspace{\abovedisplayskip}\noindent\begin{tabular}{>{$}l<{$} @{${}={}$} 
l}}
  {\end{tabular}\par\vspace{\belowdisplayskip}}


\begin{document}

    
    \begin{frame}{Sommaire}
        \tableofcontents
    \end{frame}

%%%%%%%%%%%%%%%%%%%%%%%%%%%%
 \section{Propriétés physiques et chimiques}
\begin{frame}{Rappels}
 \begin{itemize}
  \item methane -163
  \item ethane
  \item ...
 \end{itemize}

\end{frame}

\begin{frame}
\centering

\large
 $\frac{P.V}{T}=Cste$
\end{frame}

\begin{frame}
 Si V est constant et que la pression augmente ?\\
  La température augmente
\end{frame}

%TODO Chercher image
\begin{frame}
 Si T est constante et que la pression augmente ?\\
  Le volume diminue
\end{frame}

\begin{frame}
\end{frame}

\begin{frame}
 Si P est constante et que le volume augmente ?\\
  La pression diminue
\end{frame}

%%%%%%%%%%%%%%%%%%%%%%%%%%%%%%%%
\section{Spécificité du methane comme combustible}

\begin{frame}
 
\end{frame}

%%%%%%%%%%%%%%%%%%%%%%%%%%%%%%%%
\section{Liquéfaction méthane}

\begin{frame}
\end{frame}

%%%%%%%%%%%%%%%%%%%%%%%%%%%%%%%%
\section{Circuits Gas as Fuel}

%%%%%%%%%%%%%%%%
\subsection{Type A}
\begin{frame}
 
\end{frame}


%%%%%%%%%%%%%%%%
\subsection{Type C}

\begin{frame}
 
\end{frame}

\end{document}

